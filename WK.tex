% Options for packages loaded elsewhere
\PassOptionsToPackage{unicode}{hyperref}
\PassOptionsToPackage{hyphens}{url}
%
\documentclass[
]{book}
\usepackage{lmodern}
\usepackage{amssymb,amsmath}
\usepackage{ifxetex,ifluatex}
\ifnum 0\ifxetex 1\fi\ifluatex 1\fi=0 % if pdftex
  \usepackage[T1]{fontenc}
  \usepackage[utf8]{inputenc}
  \usepackage{textcomp} % provide euro and other symbols
\else % if luatex or xetex
  \usepackage{unicode-math}
  \defaultfontfeatures{Scale=MatchLowercase}
  \defaultfontfeatures[\rmfamily]{Ligatures=TeX,Scale=1}
\fi
% Use upquote if available, for straight quotes in verbatim environments
\IfFileExists{upquote.sty}{\usepackage{upquote}}{}
\IfFileExists{microtype.sty}{% use microtype if available
  \usepackage[]{microtype}
  \UseMicrotypeSet[protrusion]{basicmath} % disable protrusion for tt fonts
}{}
\makeatletter
\@ifundefined{KOMAClassName}{% if non-KOMA class
  \IfFileExists{parskip.sty}{%
    \usepackage{parskip}
  }{% else
    \setlength{\parindent}{0pt}
    \setlength{\parskip}{6pt plus 2pt minus 1pt}}
}{% if KOMA class
  \KOMAoptions{parskip=half}}
\makeatother
\usepackage{xcolor}
\IfFileExists{xurl.sty}{\usepackage{xurl}}{} % add URL line breaks if available
\IfFileExists{bookmark.sty}{\usepackage{bookmark}}{\usepackage{hyperref}}
\hypersetup{
  pdftitle={CLEANED Workbook},
  pdfauthor={Jessica Mukiri},
  hidelinks,
  pdfcreator={LaTeX via pandoc}}
\urlstyle{same} % disable monospaced font for URLs
\usepackage{longtable,booktabs}
% Correct order of tables after \paragraph or \subparagraph
\usepackage{etoolbox}
\makeatletter
\patchcmd\longtable{\par}{\if@noskipsec\mbox{}\fi\par}{}{}
\makeatother
% Allow footnotes in longtable head/foot
\IfFileExists{footnotehyper.sty}{\usepackage{footnotehyper}}{\usepackage{footnote}}
\makesavenoteenv{longtable}
\usepackage{graphicx,grffile}
\makeatletter
\def\maxwidth{\ifdim\Gin@nat@width>\linewidth\linewidth\else\Gin@nat@width\fi}
\def\maxheight{\ifdim\Gin@nat@height>\textheight\textheight\else\Gin@nat@height\fi}
\makeatother
% Scale images if necessary, so that they will not overflow the page
% margins by default, and it is still possible to overwrite the defaults
% using explicit options in \includegraphics[width, height, ...]{}
\setkeys{Gin}{width=\maxwidth,height=\maxheight,keepaspectratio}
% Set default figure placement to htbp
\makeatletter
\def\fps@figure{htbp}
\makeatother
\setlength{\emergencystretch}{3em} % prevent overfull lines
\providecommand{\tightlist}{%
  \setlength{\itemsep}{0pt}\setlength{\parskip}{0pt}}
\setcounter{secnumdepth}{5}
\usepackage{booktabs}
\usepackage[]{natbib}
\bibliographystyle{apalike}

\title{CLEANED Workbook}
\author{Jessica Mukiri}
\date{13/05/2020}

\begin{document}
\maketitle

{
\setcounter{tocdepth}{1}
\tableofcontents
}
\hypertarget{prerequisites}{%
\chapter*{Prerequisites}\label{prerequisites}}
\addcontentsline{toc}{chapter}{Prerequisites}

This lab workbook contains hands-on material supplementing the CLEANED training taught by Jessica Mukir and hosted by the Tropical Forages Program, Alliance for Bioversity Inventational and CIAT.

Recent modification of this workbook were written to augment a series of slide deck presentations developed for the training. The slides and workbook, together and along with the virtual presentations, intend to encourage the use of the CLEANED model. Use these resources to guide your learning.If you'd like, watch a video recorded from the training.

Additionally, how to , videos are integrated into the workbook chapters.The videos also serve to illustrate concepts and techniques presented in the workbook.

We hope that you will use and understand the CLEANED model.

\hypertarget{intro}{%
\chapter{Introduction to CLEANED}\label{intro}}

\hypertarget{history-of-cleaned}{%
\section{History of CLEANED}\label{history-of-cleaned}}

\hypertarget{importance-of-cleaned}{%
\section{Importance of CLEANED}\label{importance-of-cleaned}}

\hypertarget{hands-on-the-tool-the-input-section}{%
\chapter{Hands on the tool the input section}\label{hands-on-the-tool-the-input-section}}

Lets take a look at the input section.

\hypertarget{parameterization}{%
\chapter{Parameterization}\label{parameterization}}

\emph{What data do we need and where can we get it from}.

\hypertarget{outputs-and-results}{%
\chapter{Outputs and Results}\label{outputs-and-results}}

What are the different Outputs and Results.

\hypertarget{example-one-send-a-cow-report}{%
\section{Example one (Send a Cow report)}\label{example-one-send-a-cow-report}}

\hypertarget{example-two-southern-highland-report}{%
\section{Example two (Southern highland report)}\label{example-two-southern-highland-report}}

\hypertarget{applications}{%
\chapter{Applications}\label{applications}}

\hypertarget{example-one-send-a-cow-report-1}{%
\section{Example one (Send a Cow report)}\label{example-one-send-a-cow-report-1}}

\hypertarget{example-two-southern-highland-report-1}{%
\section{Example two (Southern highland report)}\label{example-two-southern-highland-report-1}}

  \bibliography{book.bib,packages.bib}

\end{document}
